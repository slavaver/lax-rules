\documentclass[../main.tex]{subfiles}

\begin{document}

\section*{Вступление}
\addcontentsline{toc}{section}{Вступление}
Во время подготовки издания 2011-2012 годов <<FIL. Книга правил игры в лакросс>>, мне стало любопытно: откуда взялись правила лакросса? Каково происхождение сегодняшних правил? Можно ли найти где-то опубликованные первоначальные правила великой игры лакросс? Уже 2014 год, а я до сих пор заинтригован этими вопросами. Проведя некоторые исследования, я нашел ответы на эти вопросы и думаю, вам тоже будет интересно их узнать.

К моему удивлению, я обнаружил, что исходные «законы» лакросса задокументированы и опубликованы. Доктор Джордж Бирс, дантист из Монреаля и Секретарь Национальной Ассоциации лакросса Канады, написал в виде документа первые правила лакросса в 1869 году. Я нашел микро-копии его буклета в архивах местной университетской библиотеки и цифровую копию на Google Books.

В книге Бира 13 глав и более 250 страниц текста, в которых подробно рассказывается о происхождении лакросса, о знании и развитии игры, а также о тренировках и проведении игр в лакросс. В приложении «Законы лакросса» перечислены 22 первоначальных правила игры.  Вы заметите, что текущая версия «FIL Книги правил» содержит 84 правила. За 148 лет игра изменилась.

Даже со всеми изменениями в игре за последние 150 лет удивительно читать оригинальную работу Бирса и видеть истоки нашей игры. Можно определить не только происхождение современных правил мужского лакросса, но и ясно понять истоки женской игры.

Если вы заинтересованы в изучении оригинальных правил лакросса или истории нашей игры, я предлагаю вам перейти в Google Books и скачать «Lacrosse: The National Game of Canada»
\footnote{\textenglish{Beers, W. G. (1869). Lacrosse: the national game of Canada. [Microform]. Montreal: Dawson Bros.}
\\ Читать на Google Books \textenglish{\url{http://books.google.com/books?id=opZJAAAAIAAJ\&printsec=frontcover}}}.
Надеюсь, вы найдете книгу Бирса столь же интересной, какой нашёл её я.

Т. Хардинг,\\ Директор отделения мужского лакросса, Международная федерация лакросса 
\end{document}
