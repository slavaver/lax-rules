\documentclass[../main.tex]{subfiles}

\begin{document}

\section{Игровое поле}
\subsection{Размеры}
\begin{enumerate}
\item Поле для игры в лакросс должно иметь форму прямоугольника с размерами 110 метров (120,3 ярдов) в длину и 60 метров (65,62 ярдов) в ширину.

\item Границы поля должны быть обозначены белыми линиями.\newline
Перпендикулярно боковым линиям через центр поля должна быть нанесена более широкая линия. Эта линия называется центральной линией.\newline
Линии границы на длинных сторонах поля называются боковыми линиями, на коротких --- лицевыми линиями. 

\item Мягкие гибкие конусы или пилоны из резины или пластика оранжевого или красного цвета, должны быть расположены: 
\begin{itemize}
  \item в каждом из четырёх углов поля; 
  \item на границе специальной зоны замен; 
  \item на каждом конце линии зоны ворот; 
  \item на конце центральной линии на противоположной стороне от зоны скамеек. 
\end{itemize}
\end{enumerate}
\noindent Конусы или пилоны должны быть установлены с внешней стороны линий границы.


\subsection{Ворота}
\begin{enumerate}
\item Ворота состоят из двух вертикальных стоек (штанг), соединенных жесткой перекладиной.\newline
Расстояние между стойками должно быть 1,83 метра (6 футов), расстояние от земли до перекладины должно быть 1,83 метра (6 футов). Все размеры внутренние.\newline
Вместе стойки ворот и перекладина называются {\sethlcolor{red}\hl{каркас}}.

\item Ворота должны быть установлены по центру поля между боковыми линиями на расстоянии 12 метров (13,12 ярдов) от лицевых линий.

\item Ворота изготавливаются из металлических труб диаметром 3,81 сантиметра (1,5 дюйма) с наружным диаметром около 5 сантиметров (2 дюймов).\newline
Они должны быть окрашены в оранжевый цвет и надежно закреплены на земле.\newline
Для обозначения плоскости ворот между стоек рисуется линия, которая называется линией ворот.

\item Стойки ворот должны быть:
\begin{itemize}
  \item либо вкопаны в землю, тогда наземный каркас не используется, или
  \item поддерживаться плоскими металлическими планками толщиной не более 1,27 сантиметров (0,5 дюйма).
\end{itemize}

Ворота должны быть сконструированы таким образом, чтобы после того как мяч зашел в ворота, он не мог вернуться обратно в поле.
\end{enumerate}

\subsection{Площадь ворот}
\begin{enumerate}
  \item Вокруг каждых ворот должен быть четко обозначенный круг.\newline
  Радиус круга 3 метра (3.28 ярда), с центром в точке посередине линии ворот.
  \item Площадь ворот это часть поля около каждых ворот ограниченная линией по кругу и включающая саму линию.
\end{enumerate}

\subsection{Сетка ворот}
\begin{enumerate}
  \item Каждые ворота должны быть оснащены сеткой пирамидальной формы, которая крепится к стойкам, перекладине и земле таким образом, чтобы задерживать мяч внутри ворот.\newline
  Размер ячейки сетки не должен превышать 3,81 сантиметра (1,4 дюйма).\newline
  К земле угол сетки крепится в точке на расстоянии 2,13 метров (7 футов) позади центра линии ворот.\newline
  Каркас вместе с сеткой называются воротами.
  \item Сетка ворот должна быть установлена таким образом, чтобы мяч мог полностью пройти через воображаемую плоскость ворот в любой точке.
  \item Сетка ворот может быть любого однородного цвета.
\end{enumerate}

\subsection{Зона ворот}
\begin{enumerate}
  \item На каждой половине поля должны быть нанесены линии между линией ворот и центральной линией на расстоянии 18 метров (19,69 ярдов) от линии ворот. Линии наносятся от одной боковой линии до другой. Эти линии называются линиями зоны ворот.
  \item Части поля между линией зоны ворот и лицевой линией (исключая сами линии) на каждой половине поля называются зонами ворот.
  \item Линии площади ворот должны выходить на 9 метров (9,84 ярда) за пределы поля в сторону, где располагаются скамьи игроков и стол секретарей. Части линий площади ворот за пределами поля обозначают границы зон скамей игроков и тренерских зон.
\end{enumerate}

\subsection{Зоны флангов}
\begin{enumerate}
  \item Параллельно боковым линиям на каждой стороне поля на расстоянии 18 метров (19,69 ярдов) от центра поля наносятся линии длинной 12,5 метров (13,67 ярдов) в каждую сторону от центральной линии. Эти линии называются линиями зоны флангов.
  \item Области поля между боковыми линиями и линиями зоны флангов и ограниченные концами линий зон флангов, исключая сами линии, называются зонами флангов.
\end{enumerate}

\subsection{Центр поля}
\begin{enumerate}
  \item Точка на центральной линии на равном расстоянии от боковых линий отмечается знаком <<Х>> или квадратом 10 на 10 сантиметров (4 на 4 дюйма) цветом, отличным от цвета центральной линии. Эта точка называется центром поля.
\end{enumerate}

\subsection{Зона замен}
\begin{enumerate}
  \item Зона замен должна быть отмечена двумя линиями на той же стороне поля что и стол секретарей.\newline
  Длина этих линий должна составлять 9 метров (9,84 ярда) и они должны быть нанесены под прямым углом к боковой линии из точек на боково линии на расстоянии 6,5 метров (7,11 ярдов) от центральной линии в сторону от поля с двух сторон. Эти линии называются линиями зоны замен.
  \item Часть боковой линии между линиями зоны замен длиной 13 метров (10 ярдов) и называется {\sethlcolor{red}\hl{гейт}}.
  \item Задняя сторона зоны замен обозначается линией параллельной боковой линии между двумя линиями зоны замен на расстоянии 4 метров (4,37 ярдов) от боковой линии.
\end{enumerate}

\subsection{Стол секретарей и скамьи игроков}
\begin{enumerate}
  \item Стол секретарей должен быть расположен на продолжении центральной линии, на расстоянии не менее 5 метров (5,47 ярдов) от боковой линии.
  \item Скамьи для соревнующихся команд должны быть расположены по обе стороны стола секретарей на расстоянии не менее 10 метров (10,94 ярдов) от него и не менее 7 метров (7,66 ярдов) от боковой линии и параллельно ей.
\end{enumerate}

\subsection{Зона скамей игроков}
\begin{enumerate}
  \item Зоны скамей игроков расположены вне игрового поля между линиями зоны замен и продолжением линий зоны ворот.\newline
  Передняя сторона зоны скамеек и зоны судейского столика обозначается линией, параллельной боковой линии на расстоянии 4 метров (4,37 ярдов) от неё.
  \item Прерогатива команды хозяев (команды, указанной в программе первой) выбрать зону скамьи игроков, которую они будут занимать во время игры.\newline
  На чемпионате мира, или аналогичном мероприятии, или в случае, когда организаторы сочтут это необходимым, команде хозяев будет выделена зона скамьи игроков слева от стола секретарей, если стоять лицом к игровому полю.
\end{enumerate}

\subsection{Тренерская зона}
\begin{enumerate}
  \item Зона, ограниченная боковой линией, линией параллельной ей на расстоянии 4 метров (4,37 ярдов), линей зоны замен, и продолжением линии зоны ворот, называется тренерской зоной.
\end{enumerate}

\subsection{Линии}
\begin{enumerate}
  \item Все линии, упомянутые в этом разделе (за исключением центральной линии и линии ворот), должны быть шириной 5 сантиметров (2 дюйма).\newline
  Центральная линия должна быть шириной 10 сантиметров (4 дюйма).\newline
  Линии ворот должны быть той же ширины, что и стойки ворот.
  \item Линии, упомянутые в этом разделе, должны быть одного цвета, и этот цвет должен контрастировать с другой разметкой, если таковая имеется на поле.
\end{enumerate}

\subsection{Штрафная скамейка}
\begin{enumerate}
  \item Штрафная скамейка состоит из двух сидений для каждой команды и находится рядом с судейским столиком.
\end{enumerate}
\end{document}
