\documentclass[../main.tex]{subfiles}

\begin{document}

\section{Оборудование}
\subsection{Мяч}

% \begin{enumerate}[start=1,label={\arabic{subsection}.\arabic*}]
\begin{enumerate}
\item Мяч должен быть изготовлен из резины белого или оранжевого цвета. Длина окружности мяча должна быть не менее 19,69 сантиметров(7,75 дюймов) и не более 20,32 сантиметров (8 дюймов).\newline
Вес мяча должен быть не менее 141,7 грамм (5 унций) и не более 148,8 граммов (5,25 унций) и при падении на твердый деревянный пол с высоты 1,83 метров (72 дюйма) должен отскакивать на высоту от 114,3 сантиметров (45 дюймов) до 124,46 сантиметров (49 дюймов).
\item Команда-хозяин предоставляет мячи на игру. Мяч, которым играли в конце игры достается выигравшей команде.

\item Команда-хозяин должна обеспечить персонал, подающий мячи, на каждой стороне поля и в каждом углу.\newline
Персонал, подающий мячи, должен носить шлем, перчатки и, в случае необходимости, ракушки.\newline
Возраст персонала, подающего мячи, должен быть не менее 10 лет.

\end{enumerate}

\subsection{Клюшка}
\begin{enumerate}
    \item Длина клюшки должна быть не менее 101,60 см (40 дюймов) и не более 106,68 см (42 дюймов) для коротких клюшек и не менее 132,08 см (52 дюймов) но не более 182,88 см (72 дюймов) для длинных клюшек.\newline 
    Команда может использовать не более четырех длинных клюшек на игровом поле за исключением клюшки номинального вратаря, игроков на штрафной скамейке и в зоне скамейки команды в те моменты игры, когда мяч «живой».
    \item Внутренние размеры головы клюшки в самом широком месте должны быть от 15,24 см (6 дюймов) до 25,40 см (10 дюймов).
% TODO добавить ссылки на правила
    \item Клюшка номинального вратаря является исключением из Правил 15.1 и 15.2. Номинальный вратарь может использовать клюшку с внутренними размерами головы в самом широком месте от 15,24 см (6 дюймов) до 38,10 см (15 дюймов) и общей длиной от 101,60 см (40 дюймов) до 182,88 см (72 дюймов).
    \item Голова клюшки должна быть изготовлены из дерева, клееной древесины, пластика, или любого другого материала, утвержденного FIL. Древко должно быть изготовлено из дерева, алюминия или любого другого материала, утвержденного FIL.\newline
    Конец древка клюшки должен быть либо цельным без острых краев, либо покрыт пластиком, резиной или лентой если древко полое для предотвращения травм.
    Использование металлических защитных колпачков запрещается.\newline
    Голова клюшки должна быть приблизительно перпендикулярна рукоятке.
    \item Голова клюшки должна быть сконструирована следующим образом:
    \begin{itemize}
        \item обе стенки должны быть изготовлены из дерева, клееной древесины, пластика или другого материала, одобренного FIL; или
        \item одна стенка должна быть изготовлена из дерева, клееной древесины, пластика или другого материала, одобренного FIL, а другая должна быть сделана путем плетения шнура от кончика головы до древка таким образом, чтобы головой клюшки нельзя было зацепиться за клюшку противника.
      \end{itemize}
      \item Высота деревянных или пластиковых стенок клюшки не должна превышать 5,08 см (2 дюймов). Если стенка изготовлена плетением, то она может быть любой высоты.
      \item На горловине клюшки может находиться защитный упор.\newline
      Упор должен быть перпендикулярен рукоятке и достаточно широким, чтобы мяч мог беспрепятственно находиться на нем.\newline
      Упор должен быть сконструирован таким образом чтобы никакая часть мяча не находилась под ним.\newline
      Защитный упор или горловина, если клюшка не оборудована защитным упором, должны располагаться на расстоянии 25,4 см (10 дюймов) от внешнего края головы.
      \item В голове клюшки и стенках должны быть отверстия для плетения и крепежа сетки.
      \item Сетка на клюшке может быть изготовлена из органической струны, сыромятной кожи, веревки, льна или синтетического материала и иметь примерно треугольную форму.
    \end{enumerate}

    \subsection{Запреты, относящиеся к клюшке}
    \begin{enumerate}
\item Игрок не должен использовать клюшку, карман которой провисает таким образом, что верхняя точка мяча, помещенного в горизонтально расположенную клюшку карманом вниз, находится ниже нижнего края боковых стенок. Это ограничение не распространяется на клюшку вратаря.
\item Игрок не должен использовать клюшку, которая сконструирована или заплетена таким образом, чтобы мяч не выходил из нее.
\item Игрок не должен использовать клюшку особой конструкции или плетения, направленной на замедление нормального и свободного выбивания мяча соперником.
\item Клюшка не должна быть модифицирована таким образом, чтобы дать игроку преимущество над соперником.
\item Длина всех концов шнурков не должна превышать 5,08 сантиметров (2 дюймов).
\item Исключено на генеральной ассамблее FIL в июне 2012.
\item Древко изменяемой длины запрещено.
\item Игрок не должен использовать клюшку, которая сконструирована или заплетена, таким образом, что она вводит в заблуждение противника, и создает впечатление что мяч находится в клюшке, когда его там нет или что мяча нет в клюшке, когда он там есть.
\end{enumerate}

\subsection{Экипировка игрока}
\begin{enumerate}
\item Все игроки обязаны носить защитные перчатки, подходящую обувь и защитный шлем, оснащенный защитной маской и ремешком для подбородка, который должен быть закреплен надлежащим образом с обеих сторон. Все игроки, включая номинального вратаря, должны использовать защиту полости рта (капу).\newline
Капа это:
\begin{itemize}
    \item Коммерческий продукт, предназначенный для формования зубов и челюстей игрока и покрытия всех зубов верхней челюсти; или
    \item Изготовленное на заказ формованное изделие изготовленное стоматологом, которое покрывает все зубы верхней челюсти; или
    \item Изготовленное на заказ формованное внутриротовое изделие изготовленное стоматологом, предназначенное для защиты игрока от травм.
  \end{itemize}
  Рекомендуется, чтобы капа была заметного цвета, т.е. она не должна быть ни белой, ни прозрачной.

  Перчатки должны быть цельными, пальцы должны быть полностью закрыты в перчатке. Игрок не может играть, если его пальцы торчат из перчатки.\newline
  Игрок может вырезать часть перчатки, закрывающую ладонь.

  Кроме того, номинальный вратарь обязан носить защитную экипировку вратаря: защиту горла, защиту грудной клетки, защиту паха. Это должны быть стандартные предметы снаряжения для полевого лакросса.

  Номинальный вратарь может носить защиту голени (футбольные щитки), компрессионное шорты или штаны для американского футбола. Компрессионные шорты или футбольные штаны можно носить как со специально предназначенными накладками, так и без них.

  За исключением клюшки вся экипировка, надеваемая вратарем, должна служить исключительно целям защиты головы и тела и не должна включать какие-либо приспособления, предоставляющие вратарю дополнительную помощь в защите ворот. Форма вратаря, майка и шорты должны носиться поверх защитной экипировки и не давать вратарю дополнительную помощь в защите ворот.

  В случае если игрок, участвующий в игровом эпизоде, теряет какой-либо обязательный элемент защитной экипировки, игра должна быть немедленно остановлена. В противном случае, судья должен дать свисток позднее, в том же порядке, что и в Правиле 82, в котором прописана техника «отложенного свистка», только в этом случае флаг не выбрасывается.

  \item Каждый игрок должен носить футболку с номером по центру спереди и сзади с использованием контрастного или готического шрифта. Номер на груди должен быть не менее 20,32 см (8 дюймов) в высоту, номер на спине должен быть не менее 25,40 см (10 дюймов) в высоту.\newline
  Номера на груди и на спине должны совпадать. Не допускается наличие двух одинаковых номеров в одной команде. Номер на майке должен состоять из одной или двух цифр, и его значение не должно превышать 99.

  Все игроки одной команды должны носить одинаковые шорты одного основного цвета команды. Те игроки, кто носит штаны, должны носить штаны того же цвета.

  \item Команда гостей должна уведомить команду хозяев о цвете маек, который они собираются использовать в игре и команда хозяев должна использовать майки контрастного цвета.
  
  AR Нападающий команды «Синих», владея мячом, бежит к воротам команды «Красных», без оппонента рядом с ним.\newline
  Он теряет обязательный элемент экипировки, совершает бросок и забивает гол.
  Решение: Гол засчитывается в выше указанных обстоятельствах, но судьи должны немедленно остановить игру, если есть опасность получения травмы атакующим игроком.
  Решение: Аналогичное решение и в случае когда отстегнулся ремешок для подбородка.
\end{enumerate}

\subsection{Запреты, относящиеся к экипировке игрока}
\begin{enumerate}
\item Ни один игрок не должен крепить камеру на шлем или форму, носить или переносить снаряжение, которое по мнению судей может навредить ему или другим игрокам.
\item Номер на майке игрока должен быть всегда отчетливо виден.
\item Специальное снаряжение, которое носит вратарь, не должно превышать стандартного снаряжения, если речь идет о защите голени, горла и грудной клетки.
\item Ни один игрок не должен носить хоккейные вратарские перчатки.
\item Любой игрок может носить спортивные штаны, но они должны быть одного командного цвета и отличного от цвета противника.
\item Если члены команды носят согревающие гетры или чулки, то они должны быть одного цвета у всей команды.
\item Игрок на поле или запасной игрок не должны носить ювелирные или пирсинг.\newline
Исключением могут быть предметы, необходимые по медицинским показаниям и/или предметы религиозного культа.\newline
Тогда они должны быть надежно приклеены к соответствующей части тела игрока таким образом, чтобы избежать их запутывания с его клюшкой, другими элементами его экипировки или другого игрока.
\end{enumerate}

\end{document}
